\sectionn{Explanation}

Order Blocks (OB) are areas of high liquidity where institutions execute large orders with minimal impact on the asset's price. The motivation behind using Order Blocks is to allow institutions to enter or exit positions discreetly, without causing significant price fluctuations that could make the market too expensive or too cheap.

These blocks are significant because they represent levels where institutions previously found value. Therefore, we often expect to see price reactions, at least when the price first returns to these regions, unless the institutions change their strategy.

Neither the SEC nor NASDAQ provides a formal definition for Order Blocks; however, according to FINRA, orders exceeding 10,000 lots and/or \$200,000 in value may be considered Order Blocks \cite{OB3}. NASDAQ's August report shows that while Order Blocks account for over 20\% of total trading volume, they represent about 0.1\% of the total number of orders \cite{OB4}.


